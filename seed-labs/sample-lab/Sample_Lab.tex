%%%%%%%%%%%%%%%%%%%%%%%%%%%%%%%%%%%%%%%%%%%%%%%%%%%%%%%%%%%%%%%%%%%%%%
%%  Copyright by <Author's Name(s)>                                 %%
%%  This work is licensed under the Creative Commons                %%
%%  Attribution-NonCommercial-ShareAlike 4.0 International License. %%
%%  To view a copy of this license, visit                           %%
%%  http://creativecommons.org/licenses/by-nc-sa/4.0/.              %%
%%%%%%%%%%%%%%%%%%%%%%%%%%%%%%%%%%%%%%%%%%%%%%%%%%%%%%%%%%%%%%%%%%%%%%

\documentclass[11pt]{article}

\usepackage[most]{tcolorbox}
\usepackage{times}
\usepackage{epsf}
\usepackage{epsfig}
\usepackage{amsmath, alltt, amssymb, xspace}
\usepackage{wrapfig}
\usepackage{fancyhdr}
\usepackage{url}
\usepackage{verbatim}
\usepackage{fancyvrb}
\usepackage{adjustbox}
\usepackage{listings}
\usepackage{color}
\usepackage{subfigure}
\usepackage{cite}
\usepackage{sidecap}
\usepackage{pifont}
\usepackage{mdframed}
\usepackage{textcomp}
\usepackage{enumitem}
\usepackage{hyperref}


% Horizontal alignment
\topmargin      -0.50in  % distance to headers
\oddsidemargin  0.0in
\evensidemargin 0.0in
\textwidth      6.5in
\textheight     8.9in 

\newcommand{\todo}[1]{
\vspace{0.1in}
\fbox{\parbox{6in}{TODO: #1}}
\vspace{0.1in}
}


\newcommand{\unix}{{\tt Unix}\xspace}
\newcommand{\linux}{{\tt Linux}\xspace}
\newcommand{\minix}{{\tt Minix}\xspace}
\newcommand{\ubuntu}{{\tt Ubuntu}\xspace}
\newcommand{\setuid}{{\tt Set-UID}\xspace}
\newcommand{\openssl} {\texttt{openssl}}

% Arrows
\newcommand{\pointleft}[1]{\reflectbox{\ding{217}} \textbf{\texttt{#1}}}
\newcommand{\pointright}[1]{\ding{217} \textbf{\texttt{#1}}}
\newcommand{\pointupleft}[1]{\reflectbox{\ding{218}} \textbf{\texttt{#1}}}

% Line numbers
\newcommand{\lineone}{\ding{192}\xspace}
\newcommand{\linetwo}{\ding{193}\xspace}
\newcommand{\linethree}{\ding{194}\xspace}
\newcommand{\linefour}{\ding{195}\xspace}
\newcommand{\linefive}{\ding{196}\xspace}
\newcommand{\linesix}{\ding{197}\xspace}
\newcommand{\lineseven}{\ding{198}\xspace}
\newcommand{\lineeight}{\ding{199}\xspace}
\newcommand{\linenine}{\ding{200}\xspace}


% Fancy headers
\pagestyle{fancy}
\lhead{\bfseries SEED Labs}
\chead{}
\rhead{\small \thepage}
\lfoot{}
\cfoot{}
\rfoot{}


\definecolor{dkgreen}{rgb}{0,0.6,0}
\definecolor{gray}{rgb}{0.5,0.5,0.5}
\definecolor{mauve}{rgb}{0.58,0,0.82}
\definecolor{lightgray}{gray}{0.90}


\lstset{%
  frame=none,
  language=,
  backgroundcolor=\color{lightgray},
  aboveskip=3mm,
  belowskip=3mm,
  showstringspaces=false,
%  columns=flexible,
  basicstyle={\small\ttfamily},
  numbers=none,
  numberstyle=\tiny\color{gray},
  keywordstyle=\color{blue},
  commentstyle=\color{dkgreen},
  stringstyle=\color{mauve},
  breaklines=true,
  breakatwhitespace=true,
  tabsize=3,
  columns=fullflexible,
  keepspaces=true,
  escapeinside={(*@}{@*)}
}

\newcommand{\newnote}[1]{
\vspace{0.1in}
\noindent
\fbox{\parbox{1.0\textwidth}{\textbf{Note:} #1}}
%\vspace{0.1in}
}


%% Submission
\newcommand{\seedsubmission}{You need to submit a detailed lab report, with screenshots,
to describe what you have done and what you have observed.
You also need to provide explanation
to the observations that are interesting or surprising.
Please also list the important code snippets followed by
explanation. Simply attaching code without any explanation will not
receive credits.}

%% Book
\newcommand{\seedbook}{\textit{Computer \& Internet Security: A Hands-on Approach}, 2nd
Edition, by Wenliang Du. See details at \url{https://www.handsonsecurity.net}.\xspace}

\newcommand{\seedisbook}{\textit{Internet Security: A Hands-on Approach}, 3rd
Edition, by Wenliang Du. See details at \url{https://www.handsonsecurity.net}.\xspace}

\newcommand{\seedcsbook}{\textit{Computer Security: A Hands-on Approach}, 3rd
Edition, by Wenliang Du. See details at \url{https://www.handsonsecurity.net}.\xspace}

\newcommand{\seedcibook}{\textit{Computer \& Internet Security: A Hands-on Approach}, 3rd
Edition, by Wenliang Du. See details at \url{https://www.handsonsecurity.net}.\xspace}

%% Videos
\newcommand{\seedisvideo}{\textit{Internet Security: A Hands-on Approach},
by Wenliang Du. See details at \url{https://www.handsonsecurity.net/video.html}.\xspace}

\newcommand{\seedcsvideo}{\textit{Computer Security: A Hands-on Approach},
by Wenliang Du. See details at \url{https://www.handsonsecurity.net/video.html}.\xspace}

%% Lab Environment
\newcommand{\seedenvironment}{This lab has been tested on our pre-built
Ubuntu 16.04 VM, which can be downloaded from the SEED website.\xspace}

\newcommand{\seedenvironmentA}{This lab has been tested on our pre-built
Ubuntu 16.04 VM, which can be downloaded from the SEED website.\xspace}

\newcommand{\seedenvironmentB}{This lab has been tested on our pre-built
Ubuntu 20.04 VM, which can be downloaded from the SEED website.\xspace}

\newcommand{\seedenvironmentC}{This lab has been tested on the SEED
Ubuntu 20.04 VM. You can download a pre-built image from the SEED website, 
and run the SEED VM on your own computer. However,
most of the SEED labs can be conducted on the cloud, and 
you can follow our instruction to create a SEED VM on the cloud.\xspace}

\newcommand{\seedenvironmentAB}{This lab has been tested on our pre-built
Ubuntu 16.04 and 20.04 VMs, which can be downloaded from the SEED website.\xspace}

\newcommand{\nodependency}{Since we use containers to set up the lab environment, 
this lab does not depend much on the SEED VM. You can do this lab
using other VMs, physical machines, or VMs on the cloud.\xspace}

\newcommand{\adddns}{You do need to add the required IP address mapping to
the \texttt{/etc/hosts} file.\xspace}






\lhead{\bfseries SEED Labs -- Sample Lab}

\begin{document}

\begin{center}
{\LARGE Sample Lab Title}
\end{center}

\vspace{0.1in}
\fbox{\parbox{6in}{\small Copyright \copyright\ 2020 \ \ by Author's Name.\\
      This work is licensed under a Creative Commons
      Attribution-NonCommercial-ShareAlike 4.0 International License. 
      If you remix, transform, or build upon the material, 
      this copyright notice must be left intact, or reproduced in a way that is reasonable to
      the medium in which the work is being re-published.}}
\vspace{0.1in}



% *******************************************
% SECTION
% ******************************************* 
\section{Overview}

Give an overview of this lab. What students will do and what 
are the learning outcome. This lab covers the following topics:
\begin{itemize}[noitemsep]
\item Topic 1
\item Topic 2
\item Topic 3
\end{itemize}


\paragraph{Reading materials.}
Provide the links to some reading materials or videos. 



\paragraph{Lab environment.} The activities in this document 
have been tested on our pre-built
Ubuntu 20.04 VM, which can be downloaded from the SEED website.  


\paragraph{Acknowledgment.} 
For people who have made contribution to this project, but 
not enough to earn an authorship, we acknowledge them here. 
Moreover, if this work is based on a funded work, you can
acknowledge the sponsor here. Here is an example:

\begin{quote}
This lab was developed with the help of Jane Doe, a graduate student
in the Department of Computer Science at Syracuse University.
The document is based on the work sponsored by the US National Science Foundation.
\end{quote}




% *******************************************
% SECTION
% ******************************************* 
\section{Environment Setup}

If the lab needs some special environment setup, talk about it here. 
For example, we may need to turn off some protection mechanisms. 
If there is nothing special, we can skip this section.



% *******************************************
% SECTION
% *******************************************
\section{Lab Tasks} 

If the description of each task is not complicated, we can 
do it in one single section, and use subsections for each
tasks. However, if each task is quite complicated, and may
contain subtasks, it is better to give each task a 
separate section. 




% *******************************************
% SECTION
% ******************************************* 
\section{Guidelines}

This is the place to provide additional materials and guidelines. We typically 
put the guidelines where it belongs, i.e., inside each tasks, but 
sometimes, guidelines may be quite long, and include a lot of background 
materials. It is better to put them here. 



% *******************************************
% SECTION
% ******************************************* 
\section{Submission}

Describe the requirement for submission. If nothing special, we can
use this standard statement. 

\begin{quote}
\seedsubmission
\end{quote}


\end{document}



