%%%%%%%%%%%%%%%%%%%%%%%%%%%%%%%%%%%%%%%%%%%%%%%%%%%%%%%%%%%%%%%%%%%%%%
%%  Copyright by Wenliang Du.                                       %%
%%  This work is licensed under the Creative Commons                %%
%%  Attribution-NonCommercial-ShareAlike 4.0 International License. %%
%%  To view a copy of this license, visit                           %%
%%  http://creativecommons.org/licenses/by-nc-sa/4.0/.              %%
%%%%%%%%%%%%%%%%%%%%%%%%%%%%%%%%%%%%%%%%%%%%%%%%%%%%%%%%%%%%%%%%%%%%%%

The above \urlisorurlsare is only accessible from inside of the virtual machine, because we
have modified the \texttt{/etc/hosts} file to map the domain
name of each URL to the IP address ({\tt 10.9.0.5}) 

You may map any domain name to a particular IP address using
\texttt{/etc/hosts}. For example, you can map
\url{http://www.example.com} to the local IP address by appending the
following entry to \texttt{/etc/hosts}:

\begin{lstlisting}[backgroundcolor=]
   127.0.0.1     www.example.com
\end{lstlisting}



