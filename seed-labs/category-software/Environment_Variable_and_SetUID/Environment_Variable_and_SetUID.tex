%%%%%%%%%%%%%%%%%%%%%%%%%%%%%%%%%%%%%%%%%%%%%%%%%%%%%%%%%%%%%%%%%%%%%%
%%  Copyright by Wenliang Du.                                       %%
%%  This work is licensed under the Creative Commons                %%
%%  Attribution-NonCommercial-ShareAlike 4.0 International License. %%
%%  To view a copy of this license, visit                           %%
%%  http://creativecommons.org/licenses/by-nc-sa/4.0/.              %%
%%%%%%%%%%%%%%%%%%%%%%%%%%%%%%%%%%%%%%%%%%%%%%%%%%%%%%%%%%%%%%%%%%%%%%

\newcommand{\commonfolder}{../../common-files}

\documentclass[11pt]{article}

\usepackage[most]{tcolorbox}
\usepackage{times}
\usepackage{epsf}
\usepackage{epsfig}
\usepackage{amsmath, alltt, amssymb, xspace}
\usepackage{wrapfig}
\usepackage{fancyhdr}
\usepackage{url}
\usepackage{verbatim}
\usepackage{fancyvrb}
\usepackage{adjustbox}
\usepackage{listings}
\usepackage{color}
\usepackage{subfigure}
\usepackage{cite}
\usepackage{sidecap}
\usepackage{pifont}
\usepackage{mdframed}
\usepackage{textcomp}
\usepackage{enumitem}
\usepackage{hyperref}


% Horizontal alignment
\topmargin      -0.50in  % distance to headers
\oddsidemargin  0.0in
\evensidemargin 0.0in
\textwidth      6.5in
\textheight     8.9in 

\newcommand{\todo}[1]{
\vspace{0.1in}
\fbox{\parbox{6in}{TODO: #1}}
\vspace{0.1in}
}


\newcommand{\unix}{{\tt Unix}\xspace}
\newcommand{\linux}{{\tt Linux}\xspace}
\newcommand{\minix}{{\tt Minix}\xspace}
\newcommand{\ubuntu}{{\tt Ubuntu}\xspace}
\newcommand{\setuid}{{\tt Set-UID}\xspace}
\newcommand{\openssl} {\texttt{openssl}}

% Arrows
\newcommand{\pointleft}[1]{\reflectbox{\ding{217}} \textbf{\texttt{#1}}}
\newcommand{\pointright}[1]{\ding{217} \textbf{\texttt{#1}}}
\newcommand{\pointupleft}[1]{\reflectbox{\ding{218}} \textbf{\texttt{#1}}}

% Line numbers
\newcommand{\lineone}{\ding{192}\xspace}
\newcommand{\linetwo}{\ding{193}\xspace}
\newcommand{\linethree}{\ding{194}\xspace}
\newcommand{\linefour}{\ding{195}\xspace}
\newcommand{\linefive}{\ding{196}\xspace}
\newcommand{\linesix}{\ding{197}\xspace}
\newcommand{\lineseven}{\ding{198}\xspace}
\newcommand{\lineeight}{\ding{199}\xspace}
\newcommand{\linenine}{\ding{200}\xspace}


% Fancy headers
\pagestyle{fancy}
\lhead{\bfseries SEED Labs}
\chead{}
\rhead{\small \thepage}
\lfoot{}
\cfoot{}
\rfoot{}


\definecolor{dkgreen}{rgb}{0,0.6,0}
\definecolor{gray}{rgb}{0.5,0.5,0.5}
\definecolor{mauve}{rgb}{0.58,0,0.82}
\definecolor{lightgray}{gray}{0.90}


\lstset{%
  frame=none,
  language=,
  backgroundcolor=\color{lightgray},
  aboveskip=3mm,
  belowskip=3mm,
  showstringspaces=false,
%  columns=flexible,
  basicstyle={\small\ttfamily},
  numbers=none,
  numberstyle=\tiny\color{gray},
  keywordstyle=\color{blue},
  commentstyle=\color{dkgreen},
  stringstyle=\color{mauve},
  breaklines=true,
  breakatwhitespace=true,
  tabsize=3,
  columns=fullflexible,
  keepspaces=true,
  escapeinside={(*@}{@*)}
}

\newcommand{\newnote}[1]{
\vspace{0.1in}
\noindent
\fbox{\parbox{1.0\textwidth}{\textbf{Note:} #1}}
%\vspace{0.1in}
}


%% Submission
\newcommand{\seedsubmission}{You need to submit a detailed lab report, with screenshots,
to describe what you have done and what you have observed.
You also need to provide explanation
to the observations that are interesting or surprising.
Please also list the important code snippets followed by
explanation. Simply attaching code without any explanation will not
receive credits.}

%% Book
\newcommand{\seedbook}{\textit{Computer \& Internet Security: A Hands-on Approach}, 2nd
Edition, by Wenliang Du. See details at \url{https://www.handsonsecurity.net}.\xspace}

\newcommand{\seedisbook}{\textit{Internet Security: A Hands-on Approach}, 3rd
Edition, by Wenliang Du. See details at \url{https://www.handsonsecurity.net}.\xspace}

\newcommand{\seedcsbook}{\textit{Computer Security: A Hands-on Approach}, 3rd
Edition, by Wenliang Du. See details at \url{https://www.handsonsecurity.net}.\xspace}

\newcommand{\seedcibook}{\textit{Computer \& Internet Security: A Hands-on Approach}, 3rd
Edition, by Wenliang Du. See details at \url{https://www.handsonsecurity.net}.\xspace}

%% Videos
\newcommand{\seedisvideo}{\textit{Internet Security: A Hands-on Approach},
by Wenliang Du. See details at \url{https://www.handsonsecurity.net/video.html}.\xspace}

\newcommand{\seedcsvideo}{\textit{Computer Security: A Hands-on Approach},
by Wenliang Du. See details at \url{https://www.handsonsecurity.net/video.html}.\xspace}

%% Lab Environment
\newcommand{\seedenvironment}{This lab has been tested on our pre-built
Ubuntu 16.04 VM, which can be downloaded from the SEED website.\xspace}

\newcommand{\seedenvironmentA}{This lab has been tested on our pre-built
Ubuntu 16.04 VM, which can be downloaded from the SEED website.\xspace}

\newcommand{\seedenvironmentB}{This lab has been tested on our pre-built
Ubuntu 20.04 VM, which can be downloaded from the SEED website.\xspace}

\newcommand{\seedenvironmentC}{This lab has been tested on the SEED
Ubuntu 20.04 VM. You can download a pre-built image from the SEED website, 
and run the SEED VM on your own computer. However,
most of the SEED labs can be conducted on the cloud, and 
you can follow our instruction to create a SEED VM on the cloud.\xspace}

\newcommand{\seedenvironmentAB}{This lab has been tested on our pre-built
Ubuntu 16.04 and 20.04 VMs, which can be downloaded from the SEED website.\xspace}

\newcommand{\nodependency}{Since we use containers to set up the lab environment, 
this lab does not depend much on the SEED VM. You can do this lab
using other VMs, physical machines, or VMs on the cloud.\xspace}

\newcommand{\adddns}{You do need to add the required IP address mapping to
the \texttt{/etc/hosts} file.\xspace}






\newcommand{\seedlabcopyright}[1]{
\vspace{0.1in}
\fbox{\parbox{6in}{\small Copyright \copyright\ {#1}\ \ by Wenliang Du.\\
      This work is licensed under a Creative Commons
      Attribution-NonCommercial-ShareAlike 4.0 International License.
      If you remix, transform, or build upon the material, 
      this copyright notice must be left intact, or reproduced in a way that is reasonable to
      the medium in which the work is being re-published.}}
\vspace{0.1in}
}






\lhead{\bfseries SEED Labs -- Environment Variable and \setuid Program Lab}


\begin{document}


\begin{center}
{\LARGE Environment Variable and \setuid Program Lab}
\end{center}

\seedlabcopyright{2006 - 2016}


% *******************************************
% SECTION
% *******************************************
\section{Overview}

The learning objective of this lab is for students to understand how
environment variables affect program and system behaviors. Environment
variables are a set of dynamic named values that can affect the way running processes
will behave on a computer. They are used by most operating systems, since
they were introduced to Unix in 1979. Although environment variables affect
program behaviors, how they achieve that is not well understood by many
programmers. As a result, if a program uses  environment
variables,  but the programmer does not know that they are used, the program
may have vulnerabilities.

In this lab, students will understand how
environment variables work, how they are propagated from parent process to
child, and how they affect system/program behaviors. We are particularly
interested in how environment variables affect the behavior of \setuid
programs, which are usually privileged programs.
This lab covers the following topics:

\begin{itemize}[noitemsep]
\item Environment variables
\item \setuid programs
\item Securely invoke external programs
\item Capability leaking
\item Dynamic loader/linker
\end{itemize}


\paragraph{Readings and videos.}
Detailed coverage of the \setuid mechanism, environment variables,
and their related security problems can be found in the following:

\begin{itemize}
\item Chapters 1 and 2 of the SEED Book, \seedbook
\item Section 2 of the SEED Lecture at Udemy, \seedcsvideo
\end{itemize}

\paragraph{Lab environment.} \seedenvironmentC





% *******************************************
% SECTION
% *******************************************
\section{Lab Tasks}

Files needed for this lab are included in \texttt{Labsetup.zip},
which can be downloaded from the lab's website. 


% -------------------------------------------
% SUBSECTION
% -------------------------------------------
\subsection{Task 1: Manipulating Environment Variables}

In this task, we study the commands that can be used to set and unset
environment variables. We are using Bash in the seed account. The default
shell that a user uses is set in  the {\tt /etc/passwd} file (the last
field of each entry). You can change this to another shell program using
the command {\tt chsh} (please do not do it for this lab). Please
do the following tasks:

\begin{itemize}
\item Use {\tt printenv} or {\tt env} command to print out the
environment variables. If you are interested in some particular
environment variables, such as {\tt PWD}, you can use {\tt "printenv PWD"}
or {\tt "env | grep PWD"}.


\item Use {\tt export} and {\tt unset} to set or unset environment
variables. It should be noted that
these two commands are not separate programs; they are two of
the Bash's internal commands (you will not be able to find them
outside of Bash).

\end{itemize}





% -------------------------------------------
% SUBSECTION
% -------------------------------------------
\subsection{Task 2: Passing Environment Variables from Parent Process to Child Process}


In this task, we study how a child process gets
its environment variables from its parent. In Unix,
{\tt fork()} creates a new process by duplicating the calling process.
The new process, referred to as the child, is an exact duplicate of the calling
process, referred to as the parent; however, several things
are not inherited by the child (please see the manual of {\tt fork()} by
typing the following command: {\tt man fork}). In this task,
we would like to know whether the parent's environment variables
are inherited by the child process or not.


\paragraph{Step 1.} Please compile and run the following
program, and describe your observation. The program
can be found in the \texttt{Labsetup} folder; it can be 
compiled using \texttt{"gcc myprintenv.c"}, which 
will generate a binary called \texttt{a.out}. 
Let's run it and save the output into a file 
using \texttt{"a.out > file"}.

\begin{lstlisting}[language=C, caption={\texttt{myprintenv.c}}]
#include <unistd.h>
#include <stdio.h>
#include <stdlib.h>

extern char **environ;
void printenv()
{
  int i = 0;
  while (environ[i] != NULL) {
     printf("%s\n", environ[i]);
     i++;
  }
}

void main()
{
  pid_t childPid;
  switch(childPid = fork()) {
    case 0:  /* child process */
      printenv();          (*@\ding{192}@*)
      exit(0);
    default:  /* parent process */
      //printenv();        (*@\ding{193}@*)
      exit(0);
  }
}
\end{lstlisting}

\paragraph{Step 2.} Now comment out the {\tt printenv()} statement in
the child process case (Line \ding{192}),
and uncomment the {\tt printenv()} statement in
the parent process case (Line \ding{193}).
Compile and run the code again, and describe your
observation. Save the output in another file.


\paragraph{Step 3.} Compare the difference of these two files using
the {\tt diff} command. Please draw your conclusion.



% -------------------------------------------
% SUBSECTION
% -------------------------------------------
\subsection{Task 3: Environment Variables and {\tt execve()}}

In this task, we study how environment variables are affected
when a new program is
executed via {\tt execve()}. The
function {\tt execve()} calls a system call to load
a new command and execute it; this function never returns.
No new process is created; instead, the calling
process's text,  data, bss, and stack  are overwritten by that of
the program loaded. Essentially, {\tt execve()} runs the new program inside
the calling process. We are interested in what happens to the
environment variables; are they automatically inherited by the
new program?


\paragraph{Step 1.} Please compile and run the following
program, and describe your observation. This program simply executes
a program called \texttt{/usr/bin/env}, which prints out the environment
variables of the current process.

\begin{lstlisting}[language=C, caption=\texttt{myenv.c}]
#include <unistd.h>

extern char **environ;
int main()
{
  char *argv[2];

  argv[0] = "/usr/bin/env";
  argv[1] = NULL;
  execve("/usr/bin/env", argv, NULL);    (*@\ding{192}@*)

  return 0 ;
}
\end{lstlisting}


\paragraph{Step 2.} Change the invocation of {\tt execve()}
in Line \ding{192} to the following; describe your observation.

\begin{lstlisting}[language=C]
execve("/usr/bin/env", argv, environ);
\end{lstlisting}


\paragraph{Step 3.} Please draw your conclusion regarding how the new
program gets its environment variables.



% -------------------------------------------
% SUBSECTION
% -------------------------------------------
\subsection{Task 4: Environment Variables and {\tt system()}}

In this task, we study how environment variables are affected
when a new program is executed via the {\tt system()} function. This
function is used to execute a command, but unlike
{\tt execve()}, which directly executes a command, {\tt system()}
actually executes \texttt{"/bin/sh -c command"}, i.e., it
executes {\tt /bin/sh}, and asks the shell to execute the command.

If you look at the implementation of the {\tt system()} function, you will
see that it uses {\tt execl()} to execute {\tt /bin/sh}; {\tt execl()}
calls {\tt execve()}, passing to it the environment variables array.
Therefore, using {\tt system()}, the environment variables of the calling process
is passed to the new program {\tt /bin/sh}. Please compile and run the following program
to verify this.


\begin{lstlisting}[language=C]
#include <stdio.h>
#include <stdlib.h>

int main()
{
  system("/usr/bin/env");
  return 0 ;
}
\end{lstlisting}



% -------------------------------------------
% SUBSECTION
% -------------------------------------------
\subsection{Task 5: Environment Variable and \setuid Programs}

{\tt Set-UID} is an important security mechanism in Unix operating systems.
When a \setuid program runs, it assumes the owner's privileges. For
example, if the program's owner is root, when anyone runs this
program, the program gains the root's privileges during its execution. 
{\tt Set-UID} allows us to do many interesting things, but since 
it escalates the user's privilege, it is quite risky. Although the
behaviors of \setuid programs are decided by their program logic, not by
users, users can indeed affect the behaviors via environment variables.
To understand how \setuid programs are affected, let us first figure out
whether environment variables are inherited by the \setuid program's
process from the user's process.


\paragraph{Step 1.} Write the following program that can print out all
the environment variables in the current process.

\begin{lstlisting}[language=C]
#include <stdio.h>
#include <stdlib.h>

extern char **environ;
int main()
{
  int i = 0;
  while (environ[i] != NULL) {
    printf("%s\n", environ[i]);
    i++;
  }
}
\end{lstlisting}


\paragraph{Step 2.} Compile the above program, change its ownership to {\tt
root}, and make it a \setuid program.

\begin{lstlisting}
// Assume the program's name is foo
$ sudo chown root foo
$ sudo chmod 4755 foo
\end{lstlisting}



\paragraph{Step 3.} In your shell (you need to be in a normal user account,
not the {\tt root} account), use the {\tt export} command to set the
following environment variables (they may have already exist):

\begin{itemize}[noitemsep]
\item {\tt PATH}
\item {\tt LD\_LIBRARY\_PATH}
\item {\tt ANY\_NAME} (this is an environment variable defined by you, so
pick whatever name you want).
\end{itemize}


These environment variables are set in the user's shell process.
Now, run the \setuid program from Step 2 in your shell. After you type the
name of the program in your shell, the shell forks a child process,
and uses the child process to run the program. Please check whether all the
environment variables you set in the shell process (parent) get into
the \setuid child process.  Describe your observation. If there are
surprises to you, describe them.



% -------------------------------------------
% SUBSECTION
% -------------------------------------------
\subsection{Task 6: The PATH Environment Variable and \setuid Programs}

Because of the shell program invoked, calling {\tt system()}
within a \setuid program is quite dangerous. This is because the actual behavior of the
shell program can be affected by environment variables, such
as {\tt PATH}; these environment variables are provided by the user, who
may be malicious.  By changing these variables, malicious users can control
the behavior of the {\tt Set-UID} program. In {\tt Bash}, you can
change the {\tt PATH} environment variable in the following way (this example
adds the directory {\tt /home/seed} to the beginning of the {\tt PATH} environment variable):


\begin{lstlisting}
$ export PATH=/home/seed:$PATH
\end{lstlisting}



The \setuid program below is supposed to execute the {\tt /bin/ls} command;
however, the programmer only uses the relative path for the {\tt ls}
command, rather than the absolute path:

\begin{lstlisting}[language=C]
int main()
{
  system("ls");
  return 0;
}
\end{lstlisting}

Please compile the above program, change its owner to {\tt root}, and
make it a \setuid program.  Can you get this \setuid program to
run your own malicious code, instead of
{\tt /bin/ls}?  If you can, is your malicious code running with the root privilege?
Describe and explain your observations. \\


\paragraph{Note:}
The \texttt{system(cmd)} function executes the \texttt{/bin/sh} program first, and then
asks this shell program to run the \texttt{cmd} command.
In Ubuntu 20.04 (and several versions before), \texttt{/bin/sh} is actually
a symbolic link pointing to \texttt{/bin/dash}.
This shell program has a countermeasure that prevents itself
from being executed in a \setuid process.  Basically, if \texttt{dash} detects that it is
executed in a \setuid process, it immediately changes the effective user ID to the process's
real user ID, essentially dropping the privilege.


Since our victim program is a \setuid program,
the countermeasure in \texttt{/bin/dash} can prevent our attack.
To see how our attack works without such a countermeasure,
we will link \texttt{/bin/sh} to another shell that does not
have such a countermeasure.  We have installed a shell program
called \texttt{zsh} in our Ubuntu 20.04 VM. We use the following
commands to link \texttt{/bin/sh} to \texttt{/bin/zsh}:

\begin{lstlisting}
$ sudo ln -sf /bin/zsh /bin/sh
\end{lstlisting}




% -------------------------------------------
% SUBSECTION
% -------------------------------------------
\subsection{Task 7: The {\tt LD\_PRELOAD} Environment
Variable and \setuid Programs}


In this task, we study how \setuid programs deal
with some of the environment variables.
Several environment variables, including {\tt LD\_PRELOAD},
{\tt LD\_LIBRARY\_PATH}, and other {\tt LD\_*} influence the
behavior of dynamic loader/linker.
A dynamic loader/linker is the part of an operating system (OS) that
loads (from persistent storage to RAM) and links the shared libraries
needed by an executable at run time.

In Linux, {\tt ld.so} or {\tt ld-linux.so}, are the dynamic
loader/linker (each for different types of binary).
Among the environment variables that affect their behaviors,
{\tt LD\_LIBRARY\_PATH} and {\tt LD\_PRELOAD} are the two
that we are concerned in this lab.
In Linux, {\tt LD\_LIBRARY\_PATH} is a colon-separated set
of directories where libraries should be searched for first, before the
standard set of directories.
{\tt LD\_PRELOAD} specifies a list of additional, user-specified, shared libraries to
be loaded before all others. In this task, we will only
study {\tt LD\_PRELOAD}.


\paragraph{Step 1.}
First, we will see how these environment variables influence the
behavior of dynamic loader/linker when running a normal program.
Please follow these steps:


\begin{enumerate}
  \item Let us build a dynamic link library. Create the following program,
  and name it {\tt mylib.c}. It basically overrides the {\tt sleep()} function
  in {\tt libc}:
\begin{lstlisting}[language=C]
#include <stdio.h>
void sleep (int s)
{
  /* If this is invoked by a privileged program,
     you can do damages here!  */
  printf("I am not sleeping!\n");
}
\end{lstlisting}

  \item We can compile the above program using the following commands (in the
  {\tt -lc} argument, the second character is $\ell$):
\begin{lstlisting}
$ gcc -fPIC -g -c mylib.c
$ gcc -shared -o libmylib.so.1.0.1 mylib.o -lc
\end{lstlisting}



  \item Now, set the {\tt LD\_PRELOAD} environment variable:
\begin{lstlisting}
$ export LD_PRELOAD=./libmylib.so.1.0.1
\end{lstlisting}

  \item Finally, compile the following program {\tt myprog}, and
  in the same directory as the above dynamic link library {\tt
  libmylib.so.1.0.1}:
\begin{lstlisting}[language=C]
/* myprog.c */
#include <unistd.h>
int main()
{
  sleep(1);
  return 0;
}
\end{lstlisting}
\end{enumerate}


\paragraph{Step 2.}
After you have done the above, please run {\tt myprog} under the following
conditions, and observe what happens.

  \begin{itemize}
  \item Make {\tt myprog} a regular program, and run it as a normal user.
  \item Make {\tt myprog} a \setuid root program, and run it as a normal user.
  \item Make {\tt myprog} a \setuid root program, export the {\tt LD\_PRELOAD}
  environment variable again in the root account and run it.

  \item Make {\tt myprog} a \setuid user1 program (i.e., the owner is user1, which
        is another user account), export the {\tt LD\_PRELOAD} environment variable
	again in a different user's account (not-root user) and run it.
  \end{itemize}


\paragraph{Step 3.}
You should be able to observe different behaviors in the scenarios
described above, even though you are running the same program.  You need
to figure out what causes the difference. Environment variables
play a role here. Please design an experiment to figure out the
main causes, and explain why the behaviors in Step 2 are
different. (Hint: the child process
may not inherit the {\tt LD\_*} environment variables).



\subsection{Task 8: Invoking External Programs Using {\tt system()} versus {\tt execve()}}

Although {\tt system()} and {\tt execve()} can both be used to run new
programs, {\tt system()} is quite dangerous if used in a privileged
program, such as \setuid programs. We have seen how the PATH environment
variable affect the behavior of {\tt system()}, because the variable
affects how the shell works. {\tt execve()} does not have the problem,
because it does not invoke shell. Invoking shell has another dangerous
consequence, and this time, it has nothing to do with environment
variables.  Let us look at the following scenario.


Bob works for an auditing agency, and he needs to investigate a company for a suspected
fraud. For the investigation purpose, Bob needs to be able to read
all the files in the
company's \unix system; on the other hand, to protect the integrity of the system,
Bob should not be able to modify any file.
To achieve this goal, Vince, the superuser of the system,
wrote a special set-root-uid program (see
below), and then gave the executable permission to Bob. This program requires
Bob to type a file name at the command line, and then it will
run {\tt /bin/cat} to display the specified file. Since the program is running
as a root, it can display any file Bob specifies. However, since the program
has no write operations, Vince is very sure that Bob cannot use this special program
to modify any file.

\begin{lstlisting}[language=C, caption=\texttt{catall.c}]
int main(int argc, char *argv[])
{
  char *v[3];
  char *command;

  if(argc < 2) {
    printf("Please type a file name.\n");
    return 1;
  }

  v[0] = "/bin/cat"; v[1] = argv[1]; v[2] = NULL;
  command = malloc(strlen(v[0]) + strlen(v[1]) + 2);
  sprintf(command, "%s %s", v[0], v[1]);

  // Use only one of the followings.
  system(command);
  // execve(v[0], v, NULL);

  return 0 ;
}
\end{lstlisting}


\paragraph{Step 1:} Compile the above program, make it a root-owned
\setuid program. The program will use
{\tt system()} to invoke the command.
If you were Bob, can you compromise the integrity of the system? For example,
can you remove a file that is not writable to you?


\paragraph{Step 2:} Comment out the {\tt system(command)} statement, and
uncomment the {\tt execve()} statement; the program
will use {\tt execve()} to invoke the command. Compile the program, and
make it a root-owned \setuid.
Do your attacks in Step 1 still work? Please describe and explain your observations.



% -------------------------------------------
% SUBSECTION
% -------------------------------------------
\subsection{Task 9: Capability Leaking}

To follow the Principle of Least Privilege, \setuid programs often
permanently relinquish their root privileges if such privileges are not
needed anymore. Moreover, sometimes, the program needs to hand over its
control to the user; in this case, root privileges must be revoked.
The {\tt setuid()} system call can be used to revoke the privileges.
According to the manual, ``\texttt{setuid()} sets the effective user ID of
the calling process. If the effective UID of the caller is root, the real
UID and saved set-user-ID are also set''. Therefore, if a \setuid program
with effective UID 0 calls \texttt{setuid(n)}, the process will become a
normal process, with all its UIDs being set to \texttt{n}.

When revoking the privilege, one of the common mistakes is capability
leaking. The process may have gained some privileged capabilities when it
was still privileged; when the privilege is downgraded, if the program
does not clean up those capabilities, they may still be accessible by the
non-privileged process. In other words, although the effective user ID of
the process becomes non-privileged, the process is still privileged because
it possesses privileged capabilities.

Compile the following program, change its owner to root, and
make it a \setuid program. Run the program as a normal user.
Can you exploit the capability leaking vulnerability in this program?
The goal is to write to the {\tt /etc/zzz} file as a normal user.

\begin{lstlisting}[language=C, caption=\texttt{cap\_leak.c}]
void main()
{
  int fd;
  char *v[2];

  /* Assume that /etc/zzz is an important system file,
   * and it is owned by root with permission 0644.
   * Before running this program, you should create
   * the file /etc/zzz first. */
  fd = open("/etc/zzz", O_RDWR | O_APPEND);
  if (fd == -1) {
     printf("Cannot open /etc/zzz\n");
     exit(0);
  }

  // Print out the file descriptor value
  printf("fd is %d\n", fd);

  // Permanently disable the privilege by making the
  // effective uid the same as the real uid
  setuid(getuid());

  // Execute /bin/sh
  v[0] = "/bin/sh"; v[1] = 0;
  execve(v[0], v, 0);
}
\end{lstlisting}



% *******************************************
% SECTION
% *******************************************
\section{Submission}

%%%%%%%%%%%%%%%%%%%%%%%%%%%%%%%%%%%%%%%%

You need to submit a detailed lab report, with screenshots,
to describe what you have done and what you have observed.
You also need to provide explanation
to the observations that are interesting or surprising.
Please also list the important code snippets followed by
explanation. Simply attaching code without any explanation will not
receive credits.

%%%%%%%%%%%%%%%%%%%%%%%%%%%%%%%%%%%%%%%%


\end{document}
