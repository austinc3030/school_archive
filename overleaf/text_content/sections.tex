\section{A Viable Attack Vector?}
The authors set out to determine whether an attacker, given ample time to inspect and replicate the charging hub, create a device that could analyze the power consumption of the connected smartphone to infer: websites visited by the device, keystrokes or pin codes entered into the device, and whether the device was receiving a phone call. There is precedent for this type of an attack already common in our daily lives; the use of credit card skimmers in everyday items such as gas pumps, point-of-sale terminals, and ATM machines have proven to be a successful means of exploiting a user's data as the attack is imperceptible at the time of exploitation. It is when the bank statement comes that the victim realizes their information has been compromised. 

The authors found that the attack was technically possible by building a device that measured the current of the smartphone as it was charging. By applying machine learning to this dataset, an attacker could reasonably infer which websites the device visited, what keystrokes were made on the device, and whether the phone received a phone call while it was charging. Their research concludes that utilizing this device and machine learning, an attacker could greatly improve the success rate beyond randomly guessing.

\section{Mitigation Techniques}
\subsection{Hardware Based}
The authors next sought to develop a hardware solution be developed to mitigate the inference of information by the attacker's device. Their mitigation device, nicknamed "The Scrambler" consisted of a micro controller with built in RAM, several MOSFETs, and several load resistors. Each of the load resistors was connected via a MOSFET in such a way that when the MOSFET was activated, the resistor was placed into the charging circuit, thus increasing the current drawn from the charger. Each MOSFET was controlled by the micro controller which was flashed with code that loaded a random value from RAM to generate a random 10-bit bit string. Each bit of the bit string corresponded to each MOSFET where a 0 indicated the MOSFET was OFF, and a 1 indicating the MOSFET was ON. The micro controller would generate a new bit string at a rate of 1kHz leading to a randomized number of resistors being placed into the charging circuit. This allowed the scrambler to fluctuate the current draw from the charger between 0ma and 1488mA. \cite{10.1145/3374664.3375732}

\subsection{Software Based}
For their software mitigation mechanism, the authors tried several different experiments to determine an adequate method for increasing power consumption of the smartphone through software and concluded that CPU utilization alone was not sufficient in causing the power consumption spikes desired for a successful mitigation mechanism. To obtain the desired effects, the authors ultimately decided to use Image Manipulation Operations as this required both CPU and RAM utilization as well as allowed for fine tuning of several variables to tailor the power consumption of the device quite accurately.

Their mechanism employed an algorithm that could generate an image of any arbitrary dimension. The algorithm would then perform random rotations on the image and then write the image to memory. After writing the image to memory, the algorithm would not only release but also clean up the memory used by the image in preparation for the next iteration. The authors found this algorithm introduced several variables, including the image dimensions, the computations on the image, and the number of read from and write to memory operations, leading to the authors being able to finely tune their mechanism to have the desired effect on the charging current of the smartphone.

\section{Testing Methodology}
The devices used by the authors for their experiment were a Samsung Galaxy S5 and a Samsung Galaxy S6. Both devices were running a stock Android 6.0.1 image with the default configuration and applications installed. Both devices had an active WiFi-internet connection. The authors installed two applications on each device, the StayAlive application meant to prevent the device from dimming the screen and going into any sleep modes. The second application was a custom Android application that collected metrics from the Android OS for correlation between actions on the victim device and the readings from the malicious charging hub.

The authors employed machine learning in their data analysis for both the unmitigated and mitigated testing. In the first scenario, with no mitigation, the device was connected directly to the malicious charging hub. In the second scenario, the hardware based mitigation, the device was connected to the malicious charging hub via the hardware mitigation device. And for the software based mitigation, the device was connected directly to the malicious charging hub but also had the author's software based mitigation mechanism installed.

The authors developed three test cases to test in the unmitigated and mitigated portions of their research. The first was the website inference test. The authors compiled a list of the top 50 websites as ranked by Alexa in 2017 and utilized a native Android application to launch each website in the Chrome browser, allow the site to load for 15 seconds, close the site, and begin loading the next site.

For the keystroke inference test, the authors used an application similar to the built in Android PIN lock screen. Users held the device in their left hand in portrait orientation while using their right index finger to press each key in any random order. In total, the authors collected data on 2,400 key presses. 

For the call detection test, the authors recorded and analyzed the power consumption of the device while at idle as well as while the device was ringing. The testing consisted of twenty phone calls using each of the two distinctly different ringtones. 

\section{Successful Mitigation}
The authors determined that the mitigation techniques, both hardware based and software based, were successful in reducing an attacker's success rate. The threshold used by the authors to determine the successful mitigation of the attack was based on the success rate of an attacker randomly guessing. 

\subsection{Hardware Based}
During the website inference test, the authors found that using only two bits on The Scrambler, the attacker's success rate was reduced by a factor of roughly four. When utilizing eight bits on The Scrambler, the attacker's success rate crossed below that of random guessing.

During the keystroke inference test, the authors found that utilizing eight bits on The Scrambler was sufficient in reducing the attacker's success rate to below that of random guessing.

During the phone call inference test, the authors found that utilization of even a single bit on The Scrambler was effective at reducing the attacker's success rate to below random guessing. 

\subsection{Software Based}
For all three tests, the authors noted the largest image size tested was 2000x2000 pixels.

During the website inference test, the authors found that any arbitrary image size was effective at reducing an attacker's success rate to close to the success rate of randomly guessing.

During the keystroke inference test, the authors found that the large image size tested of 2000x2000 pixels was sufficient in reducing the attacker's success rate to below randomly guessing.

During the phone call inference test, the authors found that any use of the software mitigation mechanism was sufficient in reducing the attacker's success rate blow random guesses.

\section{Further Testing}
The authors also conducted the following test: if an attacker was aware of the mitigation techniques employed, would the mitigation techniques still be sufficient in thwarting the attack? The authors found that while the attacker's success rate did increase given the additional knowledge, the mitigation techniques they employed were still effective and noted the attacker's success rate did not notably approach the results from the unmitigated attacks. 

\section{Mitigation Drawbacks}
\label{mitigation_drawbacks}
The authors found that, while both mitigation techniques were successful in reducing the attacker's success rate below the rate of randomly guessing, they did find that the mitigation techniques negatively impacted the charging speed and power consumption of the device. As the power consumption was increased during their mitigation, the authors also noted the environmental impacts of the mitigation in terms of CO2 emissions.

\section{Limitations of the Research Study}
\label{limitations_of_the_research_study}
The authors focused on the old standard for USB charging devices. The old protocol relied on a constant 5v supply and a variable current, often up to 1500mA. Newer protocols such as Power Delivery and Fast Charging utilize variable voltages up to 20v along with variable current. While this allows a device to charge faster, this would complicate their research as the theoretical attacker's device would need to monitor not only current but also voltage. It would be interesting to see if the same type of analysis could be applied to infer information about the connected device. I would suspect this might make it more difficult as there would be a second independent variable that would need to be accounted for. An analysis could be made converting voltage and current to wattage, but I would suspect this would lead to a loss of precision that this type of analysis would benefit from.

The authors also do not mention other methods of charging, such as wireless charging. At first, I would assume this type of analysis could aid an attacker in attacking a victim device, but with my limited knowledge of the technology enabling wireless charging, I imaging additional variables would be introduced such as the inductance in the wireless charging coils, proximity between transmitter and receiver, etc. 

Another limitation of the research study surrounds the implementation of power saving modes in modern smartphones. Power saving modes were implemented to attempt to squeeze more battery life out of the device by design. Given the \nameref{mitigation_drawbacks}, it would be unlikely that a manufacturer would incorporate a software mechanism to mitigate this type of attack since it leads to greater power consumption and a longer charging time. Further, the research study neglects to mention the likelihood of a manufacturer implementing the hardware based mitigation technique. The compact design of smartphones means physical space is at a premium and adding the additional circuitry would further complicate the design and manufacturing process of smartphones. Further, with additional circuitry would imply a greater cost of materials to produce the phone. Given that manufacturers are trying to maximize profit margins, it is unlikely a hardware implementation would be added as this would further reduce their profit margin on a device that is already very costly to produce.

\section{Future Research Opportunities}
Given the \nameref{limitations_of_the_research_study}, the future opportunities for further research immediately point to the investigation of other charging protocols such as Power Delivery and Fast Charging, as well as alternative modes of charging such as wireless charging. 

Another avenue that warrants further research is whether additional information could be obtained using this attack vector. Could an attacker use this method to derive specific information? Bank account details and log in credentials for sensitive websites would be immediate candidates for research. Additional research could be focused at obtaining contact lists, calendar and schedule information, and biometric data could also be of interest to an attacker. 