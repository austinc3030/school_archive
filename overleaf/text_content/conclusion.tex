\section{Conclusion} 
\label{sec:conclusion}
Using existing attack vectors present in the wild, the authors theorized a new attack vector that could prove to be a viable threat in the future. Given the absence of digital communication with a victim device, the authors theorized an attacker could use the power consumption figures while a device charges to infer key information about a connected victim device. After proving this to be a successful attack vector, the authors sought to mitigate such an attack and devised two approaches to mitigate an attacker's attempt to infer this information.

By developing a hardware based solution and a software mechanism with the aim of mitigating the attack, the authors proved that if an attacker, equipped with a device that could function and appear identically to the publicly available charging hub, attempted to deploy this attack in the wild, there would be a viable method for mitigating the attack. The authors thoroughly analyzed the positive and negative effects employing such a mitigation technique would impose on the user experience, including additional cost of a hardware device and degraded performance due to running a computationally intensive process on the victim device.

The authors' research into this newly identified attack vector have exposed the need for further research into this type of attack vector. This type of attack and the corresponding mitigation efforts could extend to other protocols such as Power Delivery and Fast Charging, as well as other mediums of charging such as wireless charging. Further, this type of research could be employed against tablets and laptops which often accompany people to the locations where these charging hubs may be available to the public. 